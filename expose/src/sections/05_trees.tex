\section{Entscheidungsbäume, Boosting und Bagging}
\textit{Matthias Volland, Annalena Gutheil}


Um die aufgezeichneten Datensätze verschiedener Gesten zu klassifizieren, ist es zunächst sinnvoll, die Amplituden aller Frequenzanteile in den Samples eines Datensatzes zunächst in der Form zu normalisieren, so dass die Amplitude des Referenztons immer einen konstanten Wert hat. Dadurch lassen sich Datensätze, die beispielsweise mit unterschiedlicher Lautstärke oder auf verschiedenen Systemen erzeugt wurden, besser miteinander vergleichen.

Die zu klassifizierenden Gesten unterscheiden sich dahingehend, dass sie die Bandbreite des Frequenzspektrums um den Referenzton in unterschiedlicher Weise erweitern bzw. vergrößern. Um eine Geste zu erkennen, werden daher die Amplituden der aufgezeichneten Frequenzen um die Amplitude der Frequenz des Referenztons im Spektrum in beiden Richtungen analysiert. Relevant für die Erkennung der Gesten sind dabei im Allgemeinen nur diejenigen Frequenzanteile um den Referenzton, deren Amplitude den Grenzwert von 10% der Amplitude des Referenztones nicht unterschreitet. In Ausnahmefällen können die Frequenzanteile einer Geste im Spektrum durch ein lokales Minimum vom Referenzton separiert sein. Daher wird zusätzlich nach einem weiteren Ausschlag im Spektrum gesucht, dessen Maximum mindestens 30% der Amplitude des Referenztons entsprechen muss.

Für die effiziente und robuste Klassifizierung der Gesten bietet es sich an, die Komplexität der generierten Eingabedaten zu verringern, um Overfitting zu vermeiden. Da der Datensatz einer Geste aus mehreren Samples mit Frequenzbildern besteht, könnte ein möglicher Ansatz darin bestehen, den zeitlichen Verlauf der Frequenzverschiebung durch einen Vektor abzubilden und anhand dessen Richtung und Länge die Gesten zu klassifizieren.

Stark vereinfacht unterscheiden sich die Gesten dahingehend, ob und auf welcher Seite eine Veränderung der Anzahl aller zum Referenzton gehörenden Frequenzanteile stattfindet. Weiterhin muss unterschieden werden, in welche Richtung sich diese Verschiebung ausbreitet. Sie kann beispielsweise von der rechten zur linken oder von der linken zur rechten Seite um den Referenzton wandern, sowie vom Referenzton beginnend in eine Richtung ausschlagen. Diese Unterscheidungsmerkmale können sehr gut durch einen Entscheidungsbaum abgebildet werden. Dazu kann die Verschiebungsrichtung als Maß dienen, um die Eingabemenge zu unterteilen und somit eine Klassifizierung zu erreichen.

Die Besonderheit von Ensemble Learning ist, dass mit Hilfe von vielen schwachen Klassifikatoren gute Klassifikationsergebnisse erziehlt werden können. Die Entscheidungen, die die verschiedenen Klassifikatoren treffen, sind dabei verschieden, was jedoch auch die Stärke dieser Methode ist, denn sie ergänzen sich gegenseitig.
